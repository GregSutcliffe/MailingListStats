\documentclass[a4paper,10pt]{book}
\usepackage[spanish]{babel}
\usepackage[latin1]{inputenc}
\usepackage{graphicx}

% COMMENTS (optional argument is author of comment)
\newcommand{\comments}[2][?]{
  \begin{quote}
    \textbf{Comment (#1):} {\em #2}
  \end{quote}
}

\begin{document}

\chapter{Creaci\'on de un transformador de extracci\'on de datos}
\label{cap:transformer}

Crear un nuevo \textit{transformer} es un proceso sencillo y estandarizado, ya que
se han invertido muchos esfuerzos para que as�sea.

Como se explicaba en el cap�ulo de ''Dise� e Implementaci�'' un \textit{transformer}
es un m�ulo especializado en analizar los datos obtenidos de un repositorio y
convertirlos en otro tipo de informaci�, bien sea en forma de gr�icos, tablas,
informes o cualquier otro formato de datos.

A fin de simplificar este tutorial, se ha decidido organizarlo en forma de \textit{paso a paso},
de esta forma ser�mucho m� sencillo y r�ido:

\begin{enumerate}
\item Por convenio, todo fichero que guarde una clase \textit{transformer} debe tener su
nombre escrito en minsculas y con el sufijo ''\_transformer.py'' como se explica
a continuaci�:\\
    \begin{center}
    $<$Nombre$>$\_transformer.py
    \end{center}
    Por ejemplo:
    \begin{center}
    line\_modifications\_transformer.py
    \end{center}
    El nombre de la clase ser�:
    \begin{center}
    line\_modifications\_transformer
    \end{center}

NOTA: Si el m�ulo no tiene el sufijo ''\_transformer.py'' la herramienta no lo reconocer�
como un \textit{transformer} y por lo tanto nunca ser�activado. Lo mismo sucede con
el nombre de la clase, debe tener el sufijo ''\_transformer'' para que funcione como
\textit{transformer}.

\item Todo fichero que guarde en su interior un \textit{transformer} debe ser almacenado
en el directorio ''pyWholine/'' de la herramienta:

\item La plantilla de c�igo que se emplea es la siguiente:

    \begin{verbatim}
    from transformer import transformer

    class my_transformer(transformer):

        def __init__(self, config_object, output_directory=""):
            self.m_config = config_object
            ... codigo de inicializacion ...

        def announce(self, data):
            path = data[0]  # Nombre de la ruta del fichero.
            file = data[1]  # Objeto 'history_file'
            ... codigo a ejecutar por fichero procesado ...


        def finish (self):
            ... codigo a ejecutar al terminar ...


    #---------------------- UNITY TESTS ----------------------

    def test():
        print "** UNITY TEST: my_transformer.py **"
        a = counter_transformer()
        ... codigo de pruebas unitarias ...
        a.finish()

    if __name__ == "__main__":
        test()

    \end{verbatim}

\item Una ver que el \textit{transformer} ha sido implementado y probado unitariamente
es necesario indicar a la factor� de \textit{transformers} para que nuestro transformador
pueda ser instanciado, por lo tanto hay que abrir el fichero ''pyWholine/transformer\_factory.py''
y escribir justo encima de esta marca:

    \begin{center}
    \begin{verbatim}
    # ------------- END OF PUT YOUR TRANSFORMER CODE -----------------------
    \end{verbatim}
    \end{center}

Hay que escribir un c�igo similar a este, sustituyendo ''my\_transformer'' por el
nombre de nuestro \textit{transformer}:

    \begin{verbatim}
    if type_of_transformer.upper() == "NOMBRE_DEL_TRANSFORMER":
        import my_transformer
        return my_transformer.my_transformer(config_object, output_directory)
    \end{verbatim}

Donde pone ''NOMBRE\_DEL\_TRANSFORMER'' siempre debe escribirse el nombre de
nuestro \textit{transformer}, pero sin olvidar que deben estar todas las letras en maysculas, jam�
debe haber espacios en blanco en dicho nombre.

\item Tambi� hay que hacer saber al objeto ''config\_manager'' que nuestro transformador
existe, por lo tanto hay que abrir el fichero ''pyWholine/config\_manager.py''
y escribir justo encima de esta marca:

    \begin{center}
    \begin{verbatim}
    #---------------------- END OF DEFAULT VALUES ----------------------
    \end{verbatim}
    \end{center}

Cada valor del fichero de configuraci� debe tener unos comentarios explicativos,
es el momento de escribirlos:

    \begin{verbatim}
    self.parameters['NOMBRE_DEL_TRANSFORMER'] = "YES"
    self.comments['NOMBRE_DEL_TRANSFORMER'] = \
    "# Description for my transformer: \n"+\
    "#  Additional description for my transformer.\n"
    \end{verbatim}

Como el lector se ha percatado, se ha puesto como valor por defecto ''YES'', esto
quiere decir que cada vez que se ejecute la herramienta, nuestro transformador se
activar�y realizar�su trabajo, si no queremos que por defecto sea as� entonces
hay que poner ''NO'' en vez de ''YES'', poni�dolo siempre en maysculas.

Hay que tener cuidado de poner en 'NOMBRE\_DEL\_TRANSFORMER' exactamente el
mismo texto que se puso en la factor� de \textit{transformers}\footnote{Por complicar
las cosas, se puede poner un texto diferente, pero el sufijo siempre debe ser
''\_transformer''} y que no tiene espacios en blanco, para que de esta forma se
 pueda instanciar nuestro transformador, con la diferencia de que aqu�el texto puede ir en minsculas.

\item Finalmente resta borrar el fichero de configuraci� de la herramienta (llamado ''wholine.conf'')
y ejecutar la herramienta, se generar�autom�icamente un nuevo fichero de configuraci�
esta vez con un valor que represente a nuestro transformador, si no aparece un par�etro nuevo llamado
'NOMBRE\_DEL\_TRANSFORMER' es se�l de que algo no se ha hecho bien. Rep�ase los pasos
nuevamente y con mucho cuidado.
\end{enumerate}

Si estos pasos no resultan del todo claros, hay a prop�ito un \textit{transformer} muy
sencillo llamado ''counter\_transformer'' que se encuentra dentro del directorio
''pyWholine/'', este transformador simplemente cuenta cuantos ficheros se han procesado,
por lo que creemos que es de f�il comprensi� para el lector.

\section{Preguntas Frecuentes}

Para aquellos puntos poco claros se enuncian respuestas a las preguntas m�
frecuentes:

\begin{itemize}
\item \textbf{Pregunta:} Deseo que mi \textit{transformer} tenga otros valores en el fichero de
configuraci�, de forma que yo pueda regular el comportamiento de mi transformador
desde el fichero de configuraci�.

\item \textbf{Respuesta:} Es simple, simplemente hay que seguir estos pasos:

\begin{enumerate}

\item Abrir el fichero ''pyWholine/config\_manager.py''
y escribir (siempre encima de esta marca):

    \begin{center}
    \begin{verbatim}
    #---------------------- END OF DEFAULT VALUES ----------------------
    \end{verbatim}
    \end{center}
Ahora se escriben las opciones deseadas con sus comentarios pertinentes y un valor
por defecto para cada una de ellas, de esta forma:

    \begin{verbatim}
    self.parameters['NOMBRE_DE_LA_OPCION'] = "VALOR POR DEFECTO DE LA OPCION"
    self.comments['NOMBRE_DE_LA_OPCION'] = \
    "# Description for my option: \n"+\
    "#  Additional description for my option.\n"
    \end{verbatim}

Cabe decir que el ''VALOR POR DEFECTO DE LA OPCION'' siempre deber�ser una
cadena de caracteres, aunque se representen nmeros.

\item Para coger el valor de la opci� desde nuestro transformador, basta
escribir en su c�igo:

    \begin{center}
    \begin{verbatim}
    value = self.m_config.get_value('NOMBRE_DE_LA_OPCION')
    \end{verbatim}
    \end{center}

Con ello conseguiremos el valor de la opci� buscada.

\item Finalmente hay que borrar el fichero de configuraci� de la herramienta
y volver a ejecutarla, la herramienta generar�un nuevo fichero de configuraci�
con los nombres de las nuevas opciones y sus valores por defecto. Debe verificarse
que as�ha sido, en caso de error habr�que repetir todos los pasos.

\end{enumerate}
\end{itemize}


\end{document}
